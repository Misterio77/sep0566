\documentclass{article}
\usepackage{amsmath}

\renewcommand{\thesubsection}{\thesection.\alph{subsection}}

\title{Lista 1 --- SEP0566}
\author{Gabriel Fontes}

\begin{document}

\maketitle

\section{Exercício 1}
\subsection{}
\subsection{}
\subsection{}

\[
	\left\{
	\begin{aligned}
		 & R = P_a Q_a + P_v Q_v \\
		 & U = Q_a Q_v
	\end{aligned}
	\right.
\]

Montando a função lagrangiana:
\[
	\begin{aligned}
		L & = U + \lambda(Ro)                            \\
		  & = (Q_a Q_v) + \lambda(R - P_a Q_a - P_v Q_v)
	\end{aligned}
\]

Tirando as derivadas parciais e as igualando à zero (pontos estacionários):
\[
	\left\{
	\begin{aligned}
		\frac{dL}{dQ_a}     & = Q_v - \lambda P_a = 0     \\
		\frac{dL}{dQ_v}     & = Q_a - \lambda P_v = 0     \\
		\frac{dL}{d\lambda} & = R - P_a Q_a - P_v Q_v = 0
	\end{aligned}
	\right.
\]

\[
	\begin{aligned}
		Q_a & = \frac{R}{2P_a} \\
		Q_v & = \frac{R}{2P_v}
	\end{aligned}
\]

Agora basta substituir \(P_v\), \(P_a\), e \(R\) com os dados fornecidos pelo exercício:
\[
	\left\{
	\begin{aligned}
		 & R = 12  \\
		 & P_a = 1 \\
		 & P_v = 3
	\end{aligned}
	\right.
\]
\[
	\begin{aligned}
		Q_a & = \frac{12}{2} = 6 \\
		Q_v & = \frac{12}{6} = 2
	\end{aligned}
\]

\subsection{}
Quando a utilidade é maximizada, a TMS é simplesmente a razão dos preços; ou seja:
\[
	\frac{P_a}{P_v} = \frac{6}{2} = \frac{1}{3}
\]

\subsection{}
Maior. A escolha que maximiza utilidade tem raio de 1 unidade de vestuário para 3 unidades de alimento. Por isso, ao ter 3 vestuários e 3 alimentos, Jane está numa situação onde trocar vestuário por alimento aumenta mais a satisfação do que o contrário, logo a TMS é maior quando comparada ao ponto ótimo.

\section{}
Um raio elasticidade-preço maior que \(1\) indica um mercado elástico. Como o raio é \(2.2\), a quantidade de vendas deve aumentar em \(44\%\).

\section{}
\subsection{}
TODO
\subsection{}
TODO

\section{}
\subsection{}

\[
	\left\{
	\begin{aligned}
		 & R = P_x X + P_y Y       \\
		 & U = \sqrt{X} + \sqrt{Y}
	\end{aligned}
	\right.
\]

Montando a função lagrangiana:
\[
	\begin{aligned}
		L & = U + \lambda(Ro)                                    \\
		  & = (\sqrt{X} + \sqrt{Y}) + \lambda(R - P_x X - P_y Y)
	\end{aligned}
\]

Tirando as derivadas parciais e as igualando à zero (pontos estacionários):
\[
	\left\{
	\begin{aligned}
		\frac{dL}{dX}       & = Y - \lambda P_x = 0   \\
		\frac{dL}{dY}       & = X - \lambda P_y = 0   \\
		\frac{dL}{d\lambda} & = R - P_x X - P_y Y = 0
	\end{aligned}
	\right.
\]

\end{document}
