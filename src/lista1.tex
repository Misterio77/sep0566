\documentclass{article}
\usepackage{amsmath}

\renewcommand{\thesubsection}{\alph{subsection}}
\renewcommand{\thesubsubsection}{\roman{subsubsection}}

\title{Lista 1 --- SEP0566}
\author{Gabriel Fontes}

\begin{document}

\maketitle

\section{Exercício 1}
\subsection{}

TODO: gráfico

\subsection{}

TODO: gráfico

\subsection{}

TODO: gráfico

\[
	\left\{
	\begin{aligned}
		 & R = P_a Q_a + P_v Q_v \\
		 & U = Q_a Q_v
	\end{aligned}
	\right.
\]

Montando a função lagrangiana:
\[
	\begin{aligned}
		L & = U + \lambda(Ro)                            \\
		  & = (Q_a Q_v) + \lambda(R - P_a Q_a - P_v Q_v)
	\end{aligned}
\]

Tirando as derivadas parciais e as igualando à zero (pontos estacionários):
\[
	\left\{
	\begin{aligned}
		\frac{dL}{dQ_a}     & = Q_v - \lambda P_a = 0     \\
		\frac{dL}{dQ_v}     & = Q_a - \lambda P_v = 0     \\
		\frac{dL}{d\lambda} & = R - P_a Q_a - P_v Q_v = 0
	\end{aligned}
	\right.
\]

\[
	\begin{aligned}
		Q_a & = \frac{R}{2P_a} \\
		Q_v & = \frac{R}{2P_v}
	\end{aligned}
\]

Agora basta substituir \(P_v\), \(P_a\), e \(R\) com os dados fornecidos pelo
exercício:
\[
	\left\{
	\begin{aligned}
		 & R = 12  \\
		 & P_a = 1 \\
		 & P_v = 3
	\end{aligned}
	\right.
\]
\[
	\begin{aligned}
		Q_a & = \frac{12}{2} = 6 \\
		Q_v & = \frac{12}{6} = 2
	\end{aligned}
\]

\subsection{}

Quando a utilidade é maximizada, a TMS é simplesmente a razão dos preços; ou
seja:
\[
	\frac{P_a}{P_v} = \frac{6}{2} = \frac{1}{3}
\]

\subsection{}

Maior. A escolha que maximiza utilidade tem raio de 1 unidade de vestuário para
3 unidades de alimento. Por isso, ao ter 3 vestuários e 3 alimentos, Jane está
numa situação onde trocar vestuário por alimento aumenta mais a satisfação do
que o contrário, logo a TMS é maior quando comparada ao ponto ótimo.

\section{Exercício 2}
Um raio elasticidade-preço maior que \(1\) indica um mercado elástico. Como o
raio é \(2.2\), a quantidade de vendas deve aumentar em \(44\%\).

\section{Exercício 3}
\[
	\begin{aligned}
		Q(Y, P) & = 2.000 + 15Y - 5.5P \\
		P       & = 150                \\
		Y       & = 15 000
	\end{aligned}
\]
\subsection{Elasticidade-preço da demanda}
\[
	\begin{aligned}
		\epsilon_{p,d} & = \frac{P}{Q(Y, P)} \cdot \frac{dQ(Y, P)}{dP} \\
		               & = \frac{P}{2000 + 15Y - 5.5P} \cdot 5.5       \\
		               & = \frac{82.5}{226175}                         \\
		               & \approx 0.000365
	\end{aligned}
\]
\subsection{Elasticidade-renda da demanda}
\[
	\begin{aligned}
		\epsilon_{p,d} & = \frac{Y}{Q(Y, P)} \cdot \frac{dQ(Y, P)}{dY} \\
		               & = \frac{Y}{2000 + 15Y - 5.5P} \cdot 15        \\
		               & = \frac{225000}{226175}                       \\
		               & \approx 0.994805
	\end{aligned}
\]

\section{Exercício 4}
\subsection{}

Os coeficientes representam o peso individual de cada umas variáveis \(Y\) no
gasto \(X1\).

\subsection{}

A idade tem um peso considerável, porém inferior ao da renda.

\subsection{}

TODO


\section{Exercício 5}
\subsection{}

\[
	\left\{
	\begin{aligned}
		 & R = P_x x + P_y y       \\
		 & U = \sqrt{x} + \sqrt{y}
	\end{aligned}
	\right.
\]

Montando a função lagrangiana:
\[
	\begin{aligned}
		L & = U + \lambda(Ro)                                    \\
		  & = (\sqrt{x} + \sqrt{y}) + \lambda(R - P_x x - P_y y)
	\end{aligned}
\]

Tirando as derivadas parciais e as igualando à zero (pontos estacionários):
\[
	\left\{
	\begin{aligned}
		\frac{dL}{dx}       & = \frac{1}{2\sqrt{x}} - \lambda P_x \\
		\frac{dL}{dy}       & = \frac{1}{2\sqrt{y}} - \lambda P_y \\
		\frac{dL}{d\lambda} & =  R - P_x x - P_y y
	\end{aligned}
	\right.
\]
\[
	\left\{
	\begin{aligned}
		\frac{1}{2\sqrt{x}} - \lambda  & = 0 \\
		\frac{1}{2\sqrt{y}} - 3\lambda & = 0 \\
		R - x - 3y                     & = 0
	\end{aligned}
	\right.
\]

Isolando o \(\lambda\) na primeira equação:
\[
	\lambda = \frac{1}{2\sqrt{x}}
\]

E substituindo na segunda:
\[
	\begin{aligned}
		3\lambda            & = \frac{1}{2\sqrt{y}} \\
		\frac{3}{2\sqrt{x}} & = \frac{1}{2\sqrt{y}} \\
		2\sqrt{x}           & = 6\sqrt{y}           \\
		\sqrt{x}            & = 3\sqrt{y}
	\end{aligned}
\]

Inserindo \(x\) na terceira equação e isolando \(y\):

\[
	\begin{aligned}
		I & = x +3y              \\
		I & = \sqrt{x}^2 +3y     \\
		I & = (3\sqrt{y})^2 + 3y \\
		I & = 12y                \\
		y & = \frac{I}{12}
	\end{aligned}
\]

Inserindo \(y\) na terceira equação e isolando \(x\):

\[
	\begin{aligned}
		I & = x + 3y         \\
		I & = x +\frac{I}{4} \\
		x & = \frac{3I}{4}
	\end{aligned}
\]

\subsection{}

\[ x = \frac{3I}{4} = \frac{300}{4} = 75 \]
\[ y = \frac{I}{12} = \frac{100}{12} = \frac{25}{3} \]

\subsection{}

A elastcidade-renda da demanda por roupas \(x/I\) é \(1/12\). Enquanto a da
demanda por alimentos é \(3/4\).

\section{Exercício 6}
\subsection{}
\begin{center}
	\begin{tabular}{|c | c | c | c|}
		\hline
		Funcionários & Produção & Produto Marginal & Produto Médio \\
		\hline
		1            & 10       & 10               & 10            \\
		\hline
		2            & 17       & 7                & 8.5           \\
		\hline
		3            & 22       & 5                & 7.33          \\
		\hline
		4            & 25       & 3                & 6.25          \\
		\hline
		5            & 26       & 1                & 5.2           \\
		\hline
		5            & 25       & -1               & 4.16          \\
		\hline
		5            & 23       & -2               & 3.285         \\
		\hline
	\end{tabular}
\end{center}

O produto médio é calculado dividindo a quantidade de produção pela quantidade
de rabalho:
\[
	\frac{q}{L}
\]

Por exemplo, para \(q = 26\) e \(L=5\):
\[
	\frac{26}{5} = 5.2
\]

O produto marginal é dado pela variação da quantidade de produção dividida pela
variação correspondente da quantidade de trabalho:
\[
	\frac{\Delta q}{\Delta L}
\]

Por exemplo, para \(q = 17\) e \(L = 2\):
\[
	\frac{17-10}{2-1} = 7
\]

\subsection{}

Sim. Ao ultrapassar 5 funcionários, o número de cadeiras passa a diminuir.

\subsection{}

Existem vários motivos possíves: incapacidade de absorver mais força de
trabalhos (ex: estações de trabalho), cultura da empresa, improdutividade,
distrações, etc.

\section{Exercício 7}

É 200. Neste ponto, a substituição de trabalho por maquinário/capital tem
\(1/4\) da eficiência quando comparada à substituição contrária.

TODO: Como saber se não é 12.5?

\section{Exercício 8}
\subsection{}

O produto marginal se dá pela divisão da variação da produção pela variação do
insumo. Portanto, para obter a função do PMA, iremos derivar \(Q\) com relação
à \(x\):

\[
	\begin{aligned}
		Q(x)             & = 6x^2 - 0.4x^3 \\
		\frac{dQ(x)}{dx} & = 12x - 1.2x^2
	\end{aligned}
\]

\subsection{}

O produto médio se dá pela divisão da produção pela quantidade de insumo. Então basta dividir \(Q\) por \(x\):

\[
	\frac{Q(x)}{x} = \frac{6x^2 - 0.4x^3}{x} = 6x - 0.4x^2
\]

\subsection{}

Pontos críticos são caracterizados por inclinação, isto é, derivada zero. Vamos
derivar e igualar para encontrar os pontos críticos. Após isso, iremos derivar
mais uma vez e observar o sinal para determinar a concavidade de \(G\) em cada
ponto crítico, e isso nos apontará se é ponto máximo ou mínimo.

Buscamos um ponto crítico onde a concavidade seja para baixo, isto é, uma raíz
de \(G'(x)\) cujo \(G''(x)\) seja negativo.

\[
	\begin{aligned}
		G'(x)        & = 0      \\
		12x - 1.2x^2 & = 0      \\
		12x          & = 1.2x^2 \\
		x_1          & = 10     \\
		x_2          & = 0
	\end{aligned}
\]

Agora verificar a concavidade em \(x=10\) e \(x=0\):

\[
	\begin{aligned}
		G''(x)  & = 12 - 2.4x \\
		G''(10) & = 12 - 24   \\
		        & = -12 < 0   \\
		G''(0)  & = 12 - 0    \\
		        & = 12 > 0
	\end{aligned}
\]

Portanto, nosso ponto \(x\) que maximiza \(Q\) é \(x = 10\).

\subsection{}

Faremos o mesmo que o anterior, mas para a função PMA. Ou seja, nossa função a
ser derivada duas vezes é a \(G'(x)\).

\[
	\begin{aligned}
		G''(x)    & = 0    \\
		12 - 2.4x & = 0    \\
		12        & = 2.4x \\
		x         & = 5
	\end{aligned}
\]

Agora verificar a concavidade em \(x=5\):

\[
	G'''(x)  = - 2.4
\]

A terceira derivada de \(G\) é sempre negativa (ou seja, G' é sempre concava
para baixo) e só temos um ponto crítico, então podemos concluir que o ponto de
máximo da PMA é em \(x=5\).

\subsection{}

Vamos inspecionar os pontos críticos e concavidade da nossa função PME:

\[
	\begin{aligned}
		PME'(x)        & = 0   \\
		(6x - 0.4x^2)' & = 0   \\
		6 - 0.8x       & = 0   \\
		x              & = 7.5
	\end{aligned}
\]

Temos apenas uma raiz. Vamos verificar a concavidade:

\[
	PME''(x) = -0.8
\]

Mais uma vez, uma concavidade sempre para baixo. Podemos então concluir que o
ponto de máximo da PME é \(x=7.5\).

\subsection{}

TODO: gráfico

\section{Exercício 9}
\subsection{}

A receita marginal é determinada pela receita \(R\) derivada com relação a
quantidade de produção \(Q\). A receita por sua vez é dada pelo preço \(P\)
multiplicado pela quantidade \(Q\).

\[
	RMg = \frac{dR}{dQ} = \frac{dPQ}{dQ} = P = 10
\]
Ou seja, produzir mais um produto irá aumentar a receita em U\$10, o mesmo
valor unitário do produto.

\subsection{}

O \(CMg\) é dado pelo preço do trabalho/insumo \(w = 10\) dividido pelo produto
marginal do trabalho, este dado pela função de produção \(Q\) derivada em
relação a quantidade de insumo \(X\).

Vamos encontrar o produto marginal do trabalho derivando \(Q\):
\[
	\begin{aligned}
		Q                      & = 10x - 0.5x^2 \\
		\implies \frac{dQ}{dX} & = 10 - x
	\end{aligned}
\]

E agora substituindo em \(CMg = w/PMg\):
\[
	\begin{aligned}
		CMg & = \frac{10}{10 - x} \\
		CMg & = 1 - \frac{x}{10}
	\end{aligned}
\]

TODO: escrever CMg em função de Q e não x (não consegui)

\subsection{}
TODO

\section{Exercício 10}
\subsection{}

\subsubsection{}
\(0.7\)

\subsubsection{}
\(0.35\)

\subsection{}

Alterar a quantidade de mão de obra afeta mais significativamente a produção do
quando comparado ao capital.

\section{Exercício 11}
\subsection{}
\subsubsection{}
0.45
\subsubsection{}
0.20
\subsubsection{}
0.30

\subsection{}
\[
	2\% * 0.45 = 1.9\%
\]

\subsection{}
\[
	-3\% * 0.30 = -0.9\%
\]

\subsection{}

Há rendimentos de escala decrescentes.

\section{Exercício 12}

\subsection{}

Zero.

\subsection{}

Se dá pelo \(Q\) dividido por \(L\), basta abaixar um de cada expoente:
\[
	PME(L) = -0.005L^2 + 0.3L
\]

\subsection{}

Se dá pela derivada de \(Q\) em relação à \(L\):

\[
	\begin{aligned}
		PMA(L) & = \frac{d}{dL}(-0.005L^3 + 0.3L^2) \\
		       & = -0.015L^2 + 0.6L
	\end{aligned}
\]


\subsection{}

Basta derivar \(PMA\) e igualar a zero para obter seus pontos críticos:

\[
	\begin{aligned}
		\frac{d}{dL}PMA & = - 0.03L + 0.6 = 0 \\
		0.03L           & = 0.6               \\
		L               & = 20
	\end{aligned}
\]

Temos \(L = 20 \) como ponto crítico.

Depois derivar novamente e analizar os sinais para encontrar onde a concavidade
é voltada para baixo:

\[
	\frac{d}{dL}PMA' = - 0.03
\]

Uma função constante sempre negativa, logo concavidade para baixo e o ponto
crítico é ponto de máximo.

Portanto, para maximizar o produto marginal, temos \(L = 20\).

\subsection{}

Basta igualar \(PMA\) e \(PME\).

\[
	\begin{aligned}
		PMA              & = PME              \\
		-0.015L^2 + 0.6L & = -0.005L^2 + 0.3L \\
		0                & = 0.01L^2 - 0.3L   \\
		0                & = L(0.01L - 0.3)   \\
		0                & = L(L - 30)
	\end{aligned}
\]

No ponto \(L=30\) se têm o produto médio máximo, adicionar mais mão de obra o
diminuirá.

A outra intersecção é quando a produção é zero, é de se esperar que a marginal
também seja zero.

\section{Exercício 13}

\[
	\begin{aligned}
		C   & = rK + wL   \\
		500 & = 20X + 25Y
	\end{aligned}
\]

\subsection{}
\[
	L = (0.5XY - 0.1X^2 - 0.05Y^2) + \lambda(20X + 25Y - 500)
\]

\subsection{}
\[
	\left\{
	\begin{aligned}
		\frac{dL}{dX}       & = 0.5Y - 0.2X + 20\lambda = 0 \\
		\frac{dL}{dY}       & = 0.5X - 0.1Y + 25\lambda = 0 \\
		\frac{dL}{d\lambda} & = 20X + 25Y - 500 = 0         \\
	\end{aligned}
	\right.
\]

\subsection{}

Começando com a terceira equação, que tem duas icógnitas:
\[
	\begin{aligned}
		20X + 25Y & = 500        \\
		20X       & = 500 - 25Y  \\
		X         & = 25 - 1.25Y
	\end{aligned}
\]

Substituir na segunda:

\[
	\begin{aligned}
		0         & = 0.5(25 - 1.25Y) - 0.1Y + 25\lambda \\
		25\lambda & = 0.1Y - 0.5(25 - 1.25Y)             \\
		25\lambda & = 0.725Y - 12.5                      \\
		\lambda   & = 0.03Y - 0.5
	\end{aligned}
\]

Substituindo na primeira:
\[
	\begin{aligned}
		0 & = 0.5Y - 0.2(25 -1.25Y) + 20(0.03Y - 0.5) \\
		0 & = 0.5Y - 5 + 0.25Y + 0.6Y - 10            \\
		0 & = 1.35Y - 15                              \\
		Y & = 11.111
	\end{aligned}
\]

Agora unindo novamente:
\[
	\begin{aligned}
		20X & = -25 * 11.1111 + 500 \\
		20X & = 500 - 277.7777      \\
		X   & = 222.222/20          \\
		X   & = 11.111
	\end{aligned}
\]
\[
	\begin{aligned}
		\lambda & = 0.03Y - 0.5        \\
		\lambda & = 0.03(11.111) - 0.5 \\
		\lambda & = -0.17
	\end{aligned}
\]

\subsection{}
Aproximadamente 11 unidades de cada um.

\subsection{}
O multiplicador \(\lambda\) representa

\section{Exercício 14}
\subsection{}
TODO
\subsection{}
TODO

\section{Exercício 15}
\subsection{}
TODO
\subsection{}
TODO
\subsection{}
TODO

\section{Exercício 16}
\subsection{}
TODO
\subsection{}
TODO
\subsection{}
TODO
\subsection{}
TODO

\section{Exercício 17}
\subsection{}
TODO
\subsection{}
TODO
\subsection{}
TODO

\section{Exercício 17}
\subsection{}
TODO
\subsection{}
TODO

\section{Exercício 18}
\subsection{}
TODO
\subsection{}
TODO

\section{Exercício 18}
TODO


% =====


\[
	p \cdot f(x_1, x_2) = P \cdot Q = RT
\]
\[
	C = x_1 \cdot w + x_2 \cdot r
\]

\[
	L = f(x_1, x_2) + \lambda(x_1 \cdot w + x_2 \cdot r - C)
\]

\[
	\left\{
	\begin{aligned}
		\frac{dL}{dx_1}     & = \frac{df}{dx_1} + \lambda \cdot w = 0 \\
		\frac{dL}{dx_2}     & = \frac{df}{dx_2} + \lambda \cdot r = 0 \\
		\frac{dL}{d\lambda} & = x_1 \cdot w + x_2 \cdot r - C  = 0
	\end{aligned}
	\right.
\]

\end{document}
