\documentclass{article}
\usepackage{amsmath}

\renewcommand{\thesubsection}{\thesection.\alph{subsection}}

\title{Lista 1 --- SEP0566}
\author{Gabriel Fontes}

\begin{document}

\maketitle

\section{Exercício 1}
\subsection{}
\subsection{}
\subsection{}

\[
	\left\{
	\begin{aligned}
		 & R = P_a Q_a + P_v Q_v \\
		 & U = Q_a Q_v
	\end{aligned}
	\right.
\]

Aplicando o método dos multiplicadores de lagrange:
\[
	\begin{aligned}
		L & = U + \lambda(Ro)                            \\
		  & = (Q_a Q_v) + \lambda(R - P_a Q_a - P_v Q_v)
	\end{aligned}
\]

Tirando as derivadas parciais:
\[
	\frac{dL}{dQ_a} = Q_v - \lambda P_a
	\implies
	\lambda_1 = \frac{Q_v}{P_a}
\]
\[
	\frac{dL}{dQ_v} = Q_a - \lambda P_v
	\implies
	\lambda_2 = \frac{Q_a}{P_v}
\]

\[
	\begin{aligned}
		\lambda_1       & = \lambda_2          \\
		\frac{Q_v}{P_a} & = \frac{Q_a}{P_v}    \\
		Qv              & = \frac{Q_a}{P_v}P_a
	\end{aligned}
\]

Substituindo na equação da renda:
\[
	\begin{aligned}
		R   & = P_a Q_a + P_v Q_v                            \\
		    & = P_a Q_a + P_v\left(\frac{Q_a}{P_v}P_a\right) \\
		    & = P_a Q_a + Q_a P_a                            \\
		    & = 2(P_a Q_a)                                   \\
		Q_a & = \frac{R}{2P_a}
	\end{aligned}
\]
\[
	\begin{aligned}
		R   & = P_a Q_a + P_v Q_v                        \\
		    & = P_a\left(\frac{R}{2P_a}\right) + P_v Q_v \\
		    & = \frac{R}{2} + P_v Q_v                    \\
		Q_v & = \frac{R}{2P_v}
	\end{aligned}
\]

Agora basta substituir \(P_v\), \(P_a\), e \(R\) com os dados fornecidos pelo exercício:
\[
	\left\{
	\begin{aligned}
		 & R = 12  \\
		 & P_a = 1 \\
		 & P_v = 3
	\end{aligned}
	\right.
\]
\[
	\begin{aligned}
		Q_a & = \frac{12}{2} = 6 \\
		Q_v & = \frac{12}{6} = 2
	\end{aligned}
\]

\subsection{}
Quando a utilidade é maximizada, a TMS é simplesmente a razão dos preços; ou seja:
\[
	\frac{P_a}{P_v} = \frac{6}{2} = \frac{1}{3}
\]

\subsection{}
Maior. A escolha que maximiza utilidade tem raio de 1 unidade de vestuário para 3 unidades de alimento. Por isso, ao ter 3 vestuários e 3 alimentos, Jane está numa situação onde trocar vestuário por alimento aumenta mais a satisfação do que o contrário, logo a TMS é maior quando comparada ao ponto ótimo.

\section{}
Um raio elasticidade-preço maior que \(1\) indica um mercado elástico. Como o raio é \(2.2\), a quantidade de vendas deve aumentar em \(44\%\).

\section{}
\subsection{}
TODO
\subsection{}
TODO

\section{}
\subsection{}
Vamos resolver de forma geométrica.

Os valores de demanda que buscamos são os valores de consumo ótimos.
Graficamente, esses valores são as coordenadas do ponto onde a curva gerada
pela função utilidade maximizada se intersecta com a reta da renda.

Para encontrar esse ponto, vamos igualar a função utilidade a uma constante
\(u\) tal que a reta da renda se cruze uma, e apenas uma, vez com a função
utilidade; ou seja, seja igual à reta tangente desta curva.

Podemos encontrar esse ponto de tangente igualando a derivada da curva (gerada
pela função de utilidade) com a derivada da reta da renda. Isso é possível
pois o valor dessas duas derivadas coincidem apenas no ponto da tangente.

Vamos começar derivando a função da curva gerada pela função \(U\):
\[
	\begin{aligned}
		U        & = \sqrt{x} + \sqrt{y}                                        \\
		\sqrt{y} & = U - \sqrt{x}                                               \\
		y        & = {(U - \sqrt{x})}^2                                         \\
		y'       & = 2(U - \sqrt{x}) \left(-\frac{1}{2} x^{\frac{-1}{2}}\right) \\
		y'       & = \frac{\sqrt{x} - U}{\sqrt{x}}
	\end{aligned}
\]

Agora vamos derivar a reta da renda:
\[
	\begin{aligned}
		I  & = x P_x + y P_y   \\
		I  & = x + 3y          \\
		y  & = \frac{I - x}{3} \\
		y' & = -\frac{1}{3}
	\end{aligned}
\]

Ótimo, agora vamos igualar as duas derivadas e encontrar \(U\) em termos de
\(x\):
\[
	\begin{aligned}
		\frac{\sqrt{x} - U}{\sqrt{x}} & = -\frac{1}{3}        \\
		3\sqrt{x} - 3 U               & = - \sqrt{x}          \\
		4\sqrt{x}                     & = 3U                  \\
		U                             & = \frac{4}{3}\sqrt{x}
	\end{aligned}
\]

Podemos igualar esse valor com a função utilidade e resovler para \(y\):
\[
	\begin{aligned}
		U                   & = U                    \\
		\frac{4}{3}\sqrt{x} & = \sqrt{x} + \sqrt{y}  \\
		\sqrt{y}            & = \frac{1}{3} \sqrt{x} \\
		y                   & = \frac{x}{9}
	\end{aligned}
\]

Encontramos uma outra equação de reta.

\end{document}
