\documentclass{article}
\usepackage{amsmath}

\renewcommand{\thesubsection}{\alph{subsection}}
\renewcommand{\thesubsubsection}{\roman{subsubsection}}

\title{Prova 1 2019 --- SEP0566}
\author{Gabriel Fontes}

\begin{document}

\maketitle

\section{}
\subsection{}
\[
	p \cdot f(x_1, x_2) = P \cdot Q = RT
\]
\[
	C = x_1 \cdot w + x_2 \cdot r
\]

\[
	L = p \cdot f(x_1, x_2) + \lambda(x_1 \cdot w + x_2 \cdot r - C)
\]

\subsection{}

\[
	\left\{
	\begin{aligned}
		\frac{dL}{dx_1}     & = \frac{dpf}{dx_1} + \lambda \cdot w = 0 \\
		\frac{dL}{dx_2}     & = \frac{dpf}{dx_2} + \lambda \cdot r = 0 \\
		\frac{dL}{d\lambda} & = x_1 \cdot w + x_2 \cdot r - C  = 0
	\end{aligned}
	\right.
\]

As derivadas parciais representam o quanto aquele determinado insumo impacta
no total da produção.

\subsection{}

O lambda está associado a restrição e é chamado de "preço sombra"; ele
representa em quanto o lucro aumentará para cada unidade adicionada na
restrição orçamentária.

\section{}
\subsection{}



\end{document}
